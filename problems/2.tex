%! TeX root: ../main.tex
x
\subsection{Question 2}

Let \( F \) be a forest on \( n \) vertices with \( c \) connected components. 
\begin{enumerate}[(a), leftmargin=1cm]
	\item Prove that \( F \) has \( n - c \) edges.
	\item Find the average degree of \( G \).
	\item Prove that the intersection of \( k \) connected subgraphs of \( F \) is either empty or a tree.
\end{enumerate}
\begin{proof} For (a) we proceed by induction on \( c \). If \( c = 1 \) then \( F \) is a tree so that \( |E(F)| = |V(F)| - 1 = n - 1 \) as needed. Now fix \( c \geq 2 \) and let \( C \) be any component of \( F \). Let \( F' = F \setminus C \). By induction, \( |E(F')| = |V(F')| - (c - 1) \). Since \( C \) is maximally connected in \( F \), \( |E(C)| = |V(C)| + 1 \) since it is a tree. Since there was no edge between \( F' \) and \( C \) (it was a component),
	\begin{align*}
		|E(F)| &= |E(F')| + |E(C)| \\
		       &= |V(F')| - (c - 1) + |V(C)| + 1 \\
		       &= |V(F)| - c = n - c,
	\end{align*}
	which completes the proof.

For (b) note that by handshaking and (a), \[ \sum_{v \in V(F)}^{} \deg v = 2 |E(F)| = 2(n - c)  \] so that the average degree of \( F \) is \(\frac{2}{n} (n-c).\)

Finally, for (c) we proceed as in (a) by induction on \( k \). The intersection of 1 connected subgraph of \( F \) must be a tree since \( F \) is a forest. Now let \( F_1 \) and \( F_2 \) be connected subgraphs of \( F \). Assume for a contradiction that \( F_1 \cap F_2 \) is not a tree \emph{and} non-empty. Since \( F_1 \cap F_2 \) is a subgraph of a forest, it is acyclic; thus \( F_1 \cap F_2 \) must not be connected, otherwise it is a tree. But then \( F_1 \cap F_2 \) is itself a forest. Using this and non-emptyness, there are vertices \( u,v \in V(F_1 \cap F_2) \) which lie in different conneted components. But then \( u,v \) lie in different connected components of \( F \), and \( u,v \in V(F_1) \) contradicts its connectivity, as needed. If \( k \geq 3 \), write \( F' = F_1 \cap F_2 \cap \cdots \cap F_{k-1}  \) and note that \( F' \) is either a tree or empty. In the first case, \( F' \) is connected so the induction hypothesis implies the result for \( F' \cap F_{k}  \); otherwise \( F' \) is null so that \( F' \cap F_{k}  \) is too. This completes the proof.
\end{proof}
